\documentclass[12 pt]{report}
\usepackage{soul}
\usepackage[normalem]{ulem}
\usepackage{fancyhdr}
\usepackage{color}
\usepackage[latin1]{inputenc}
\usepackage[T1]{fontenc}
\usepackage[francais]{babel}
\usepackage{lmodern}
\usepackage{amsmath}
\usepackage{amssymb}
\usepackage{mathrsfs}
\usepackage{graphicx}
\usepackage[rightcaption]{sidecap}
\usepackage{floatrow}
\usepackage{epstopdf}

\begin{document}
\begin{figure}
\begin{center}
\includegraphics[height=100, width=200, angle=0]{1.jpg}
\end{center}
\end{figure}
\center
\qquad
\Huge 
ISET'COM

\Huge\hrulefill

\center \Huge  Residual Energy and Density Control Aware Cluster
Head Election in Wireless Sensor Network

\Huge\hrulefill
\normalsize
\flushleft Auteur:                    

Premier Auteur: MEDDAH MAREIM      

Deuxieme Auteur:  RIM HADDAD      


\newpage
\center \Huge Abstract
\normalsize \flushleft 
Abstract—In designing the Wireless Sensor Network, the
energy consumption represents one of the most and important
factor that recent researches aim to reduce efficiently. Because of
the limited energy that sensor nodes are equipped with, and in
order to prolong the network lifetime, we introduce a new
clustering algorithm for WSN, which consists on choosing a
subset of sensor nodes to be elected as Cluster Heads. This choice
is based on three parameters; the residual energy, the density
and the distances separating the node with its neighbors. Each
node enters into a competition and makes its proper decision.
Then, the network is divided into clusters and each member is
affected to the nearest CH that has the job to gather the data of
the members of its cluster. The efficiency of our proposed
algorithm is proved comparing with some other clustering
approaches. Simulation results show that the energy dissipation
of the sensor nodes can be considerably reduced, and the network
lifetime efficiently extended.
\flushright Keywords— Wireless Sensor Networks; Residual Energy;
Cluster Head election; Density ; Neighbor Distance
\flushleft
\newpage 
\tableofcontents

\newpage
\chapter{Introduction}
\qquad
Over recent years, the importance of Wireless Sensor
Networks (WSNs) in different domains is considerably
increasing. It consists on a group of small sensor nodes
capable to detect and monitor physical conditions from the
field, and then to transmit data to the Base Station (BS) once
gathered. These nodes are very low-cost and are capable to
perform as an autonomous device, but have less storage
capacity and a limited battery power, that can be depleted very
quickly, due to the multiple tasks made, such as processing
and exchanging packets with nodes. 

\qquad 
The most important challenge in WSN is to extend the
network lifetime. This can be achieved when the energy
consumption is efficiently used, and balanced among nodes.
Many clustering algorithms proposed in the last few years
have the objective to improve this goal, and are focused on
how to regroup the sensor nodes deployed into clusters. Each
created cluster will have a Cluster Head (CH). This node,
elected in each round, collect data from the other members of
the cluster, and then send it to the BS.

\qquad 
Furthermore, to reduce energy consumption, previous
researches in WSN have been done, considering the residual
energy as parameter [1] [2] [3], taking into consideration the
communication cost, the required amount of energy needed for
a node to send a message to another one [4], or calculating a
weight for each node to be compared [3]. Also, a random
number was used in order to have a balanced CH election in
the network [5] [6]. These different parameters are used to
elect the suitable CH. But for the clustering step, in some
researches, the whole network is divided into several clusters,
choosing the number of clusters before each round of the
clustering process [7].

\qquad 
In this paper, a new distributed and competitive algorithm,
named Cluster Head Election using Residual Energy and
Density Control (CHEREDC) for wireless sensor network is
proposed. In the CH election, each node in the network enters
into an energy and density comparisons. In our algorithm,
each node must compare its Residual Energy and its density
with the Residual Energy and the density of the neighbor

nodes. In the cluster creation phase, each CH will choose the
set of member nodes to be affected to its cluster.

\qquad 
The remainder of this paper is organized as follows: Section
II discusses some of the existing clustering techniques; in
Section III the system model is presented; Section IV and V
present the proposed algorithms CHEREDC and evaluate its
performance comparing with previous algorithms. Finally, we
conclude the paper and discuss some possible future works in
Section VI.
\chapter{related work}
\qquad
Most of the previous clustering approaches have been done
aiming to balance the energy consumption of the nodes over
the network. In this section, we briefly introduce some of these
proposed algorithms.

\qquad 
In Low-energy and adaptive clustering hierarchy (LEACH)
[9], Energy-Efficient Unequal Clustering (EEUC) [5], Cluster
Head Election mechanism using Fuzzy logic (CHEF) [6] and
Multi Objectives Fuzzy Clustering Algorithm (MOFCA) [2] a
probabilistic and distributed method is used, where, the node
enters into a competition only if its random generated number
is less than a predefined threshold value Th. For [5] [6] and
[2], the competition is made by neighbors, they communicate
and decide together the best node eligible to be elected as CH
without any contribution from the BS. Therefore, this method
applied in the election of the CH differs from an algorithm to
another. In fact, in CHEF, the value of the range is static and
considered the same for each implemented node, contrary to
MOFCA [2] and EEUC [5] where it can vary, according to the
distance separation the node with the BS in [5], or depending
on the density, the remaining energy and the distance
separating the node to the BS in [2]. The non-collaboration of
the BS in the cluster head election process forces the nodes to
communicate between each other during the entire phase. This
approach did not help decreasing the relayed packets, so the
energy consumption of the nodes is increased.

\qquad 
Then, the proposed approach in Fuzzy C-Mean based
clustering algorithm (FCM) [10], considers three parameters:
the Euclidian distance to partition the network, and the
denoted Xie and Beni’s (XB) index to determine the optimal
number of cluster. The CHs elected in this algorithm are
approximately positioned at the center of their clusters and the
inter-cluster distance is not considered. Contrary to LDC-
kmean algorithm [7], Residual energy aware mobile data
gathering in WSN (REAMDC) [3] and Multi-hop Clustering
Algorithm Based on Spectral Classification for WSN (MHCA-
SC) [11] where the Euclidian distance is the only parameter
considered in the clustering step. In [7], the number of
members node assigned in the clusters is counted. This
number can vary between two values: N/K – δ and N/K + δ,
where N represents the number of nodes, K the number of
clusters chosen in the beginning of the clustering step and δ a
parameter set equal to 0%,2% or 3% of N. Next, the closest
node to the center of each cluster is elected as CH. As we
remark, the energy consumption model is not studied in [7] [9]
and [10]. These algorithms present many disadvantages in
terms of consumed energy by the deployed nodes, which are

not balanced over the network. In the recent researches, most
of the proposed algorithms consider the energy consumption
in the election of CHs because of the amount of energy that
CHs need to gather the sensed data from cluster’s member,
and send them to the BS [2] [5] [6] [12] [14].

\qquad
In [3] and [8], the election of the CHs requires a
comparison of the weight between sensor nodes in the cluster.
The node with a greater weight is the best one eligible to be
elected as CH. In Tree-cluster- based data-gathering algorithm
for industrial WSN (TCBDGA) [13], Weight-Based Trees are
established and considered as clusters using the residual
energy, the number of neighbors, and the distance to the BS as
parameters to be compared. The CHs are the nodes in each
head of the created trees.
\chapter{SYSTEM MODEL}
\section{Network Structure}
\qquad
In [3] and [8], the election of the CHs requires a
comparison of the weight between sensor nodes in the cluster.
The node with a greater weight is the best one eligible to be
elected as CH. In Tree-cluster- based data-gathering algorithm
for industrial WSN (TCBDGA) [13], Weight-Based Trees are
established and considered as clusters using the residual
energy, the number of neighbors, and the distance to the BS as
parameters to be compared. The CHs are the nodes in each
head of the created trees.
\section{Energy consumption model}
\qquad
The energy consumption model adopted here is the same
in [1]. The energy consumed to transmit and to receive l-bit
packets over a distance d is given respectively in Equation (1)
and Equation (2). Where E elec represents the degenerated
energy per bit in the transmitter and receiver models. The
parameter ε fs and ε mp denote respectively the amplifier energy
in a free space model with d 2 power loss, and in a multi-path
fading model with d 4 power loss, according to a threshold
estimation of the transmission distance d o , which can be
obtained by Equation (3).
\newpage

\begin{equation}
$$E_{TX}(l,d)=\left\lbrace\begin{array}{ll}
lE_{elec}+\epsilon_{fs}	d^2,\hspace{2cm}d<d_{o}\\
lE_{elec}+\epsilon_{mp}	d^4,\hspace{2cm}d>d_{o}\\
\end{array}\right.$$
\end{equation}
\begin{equation}
$$E_{RX}(l)=lE_{elec}
\end{equation}
\begin{equation}
$$D_{0}=\sqrt{\frac{\epsilon_{fs}}{\epsilon_{mp}}}
\end{equation}

\end{document}