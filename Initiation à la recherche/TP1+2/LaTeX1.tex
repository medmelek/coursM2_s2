\documentclass[12 pt]{article}
\usepackage{soul}
\usepackage{ulem}

\usepackage{fancyhdr}
\pagestyle{fancy}


\begin{document}

\lhead{}
\chead{}
\rhead{\bfseries Oueslati Mohamed Melek}
\lfoot{TP1 et TP1}
\cfoot{Initiation à la recherche appliquée2}
\rfoot{\thepage}
\renewcommand{\headrulewidth}{1pt}
\renewcommand{\footrulewidth}{1pt}


\title {Table des matières}
\maketitle

\section{Introduction}

\section{\underline{Introduction}}
\section{\underline{Definition}}
 \subsection{\uuline {advantages}}
 \subsection{\uuline{disadvantages}}
 
\section{\underline{conclusion}}

\newpage
\setcounter{section}{0}
\section{\underline{Introduction}}

\qquad
\flushright
Over recent years, the importance of Wireless Sensor
Networks (WSNs) in different domains is considerably
increasing. It consists on a group of small sensor nodes
capable to detect and monitor physical conditions from the
field, and then to transmit data to the Base Station (BS) once
gathered. These nodes are very low-cost and are capable to
perform as an autonomous device, but have less storage
capacity and a limited battery power, that can be depleted very
quickly, due to the multiple tasks made, such as processing
and exchanging packets with nodes.

\vspace{3cm}
\qquad
\center
The most important challenge in WSN is to extend the
network lifetime. This can be achieved when the energy
consumption is efficiently used, and balanced among nodes.
Many clustering algorithms proposed in the last few years
have the objective to improve this goal, and are focused on
how to regroup the sensor nodes deployed into clusters. Each
created cluster will have a Cluster Head (CH). This node,
elected in each round, collect data from the other members of
the cluster, and then send it to the BS.

\vspace{3cm}
\qquad
\flushleft
Furthermore, to reduce energy consumption, previous
researches in WSN have been done, considering the residual
energy as parameter [1] [2] [3], taking into consideration the
communication cost, the required amount of energy needed for
a node to send a message to another one [4], or calculating a
weight for each node to be compared [3]. Also, a random
number was used in order to have a balanced CH election in
the network [5] [6]. These different parameters are used to
elect the suitable CH. But for the clustering step, in some
researches, the whole network is divided into several clusters,
choosing the number of clusters before each round of the
clustering process [7].


\newpage
\setcounter{section}{0} 

\textcolor {blue} { \section{\ul{Introduction}}   }

\paragraph{\qquad
 Most of the previous clustering approaches have been done aiming to balance the energy consumption of the nodes over the network. In this section, we briefly introduce some of these proposed algorithms.}
\raggedleft

\paragraph{\qquad
In Low-energy and adaptive clustering hierarchy (LEACH) [9], Energy-Efficient Unequal Clustering (EEUC) [5], Cluster Head Election mechanism using Fuzzy logic (CHEF) [6] and Multi Objectives Fuzzy Clustering Algorithm (MOFCA) [2] a probabilistic and distributed method is used, where, the node enters into a competition only if its random generated number is less than a predefined threshold value Th. For [5] [6] and [2], the competition is made by neighbors, they communicate and decide together the best node eligible to be elected as CH }\centering
\vspace{3cm}

\paragraph{\qquad
The remainder of this paper is organized as follows: Section
II discusses some of the existing clustering techniques; in
Section III the system model is presented; Section IV and V
present the proposed algorithms CHEREDC and evaluate its
performance comparing with previous algorithms. Finally, we
conclude the paper and discuss some possible future works in
Section VI.
}\raggedright
\vspace{3cm}

\newpage

\section{}
\begin{tabular}{|c|c|}
\hline
paramter & value\\

\hline
N (Number of deployed nodes) & 22 \\

\hline
R (Range of nodes in meter) & 10 \\

\hline
Eelec & 50nJ/bit \\

\hline
\epsilon fs & 10pJ/bit/m  \\

\hline
\epsilon mp & 0.001pJ/bit/mp4 \\

\hline
E (Initial energy of nodes) & 1J \\

\hline
ctrPacketLenth & 2000bits \\

\hline
PacketLenth & 4000bits \\

\hline
\end{tabular}

\section{}
\begin{tabular}{|l|l}
\hline
paramter & value\\

\hline
N (Number of deployed nodes) & 22 \\

\hline
R (Range of nodes in meter) & 10 \\

\hline
Eelec & 50nJ/bit \\

\hline
\epsilon fs & 10pJ/bit/m  \\

\hline
\epsilon mp & 0.001pJ/bit/mp4 \\

\hline
E (Initial energy of nodes) & 1J \\

\hline
ctrPacketLenth & 2000bits \\

\hline
PacketLenth & 4000bits \\

\hline
\end{tabular}


\section{}
\begin{tabular}{|r|r|}
\hline
paramter & value\\

\hline
N (Number of deployed nodes) & 22 \\

\hline
R (Range of nodes in meter) & 10 \\

\hline
Eelec & 50nJ/bit \\

\hline
\epsilon fs & 10pJ/bit/m  \\

\hline
\epsilon mp & 0.001pJ/bit/mp4 \\

\hline
E (Initial energy of nodes) & 1J \\

\hline
ctrPacketLenth & 2000bits \\

\hline
PacketLenth & 4000bits \\

\hline
\end{tabular}

\newpage

\section{}
\begin{tabular}{r l}
\hline
paramter \vline&value\\
\hline
\hline
N (Number of deployed nodes) & 22 \\

R (Range of nodes in meter) & 10 \\

Eelec & 50nJ/bit \\

\epsilon fs & 10pJ/bit/m  \\

\epsilon mp & 0.001pJ/bit/mp4 \\

E (Initial energy of nodes) & 1J \\

ctrPacketLenth & 2000bits \\

PacketLenth & 4000bits \\
\hline
\end{tabular}

\newpage
\section{Tableau II}
\begin{tabular}{|c|c|c|c|}
\hline
(XBS,YBS)& FND & HNA & LND \\
\hline
((100,100)) & 1003 & 1521 & 2693 \\
\hline
(0,0) & 119 & 1009 & 2457 \\
\hline
(250,250)  & 40 & 470 & 2063 \\
\hline
\end{tabular}

\section{Tableau II separation horizontal}
\begin{tabular}{cccc}
\hline
(XBS,YBS)& FND & HNA & LND \\
\hline
\hline
((100,100)) & 1003 & 1521 & 2693 \\

(0,0) & 119 & 1009 & 2457 \\

(250,250)  & 40 & 470 & 2063 \\
\hline
\end{tabular}


\section{Tableau II separation horizontal entre colonne}
\begin{tabular}{cccc}
\hline
\begin{tabular}{c|}(XBS,YBS) \\ \hline & \\  \end{tabular}& \begin{tabular}{c|}FND\\ \hline & \\  \end{tabular}  & HNA & LND  \\
\hline
((100,100)) & 1003 & 1521 & 2693 \\
\hline
(0,0) & 119 & 1009 & 2457 \\
\hline
(250,250)  & 40 & 470 & 2063 \\
\hline
\end{tabular}



\section{Tableau II  separation vertical }
\begin{tabular}{|c|c|c|c|}
\hline
(XBS,YBS)& FND & HNA & LND \\
\hline
((100,100)) & 1003 & 1521 & 2693 \\
\hline
(0,0) & 119 & 1009 & 2457 \\
\hline
(250,250) \vline Value & 40 & 470 & 2063 \\
\hline
\end{tabular}

\newpage 

\section{\ul{Exercise 1}}


\begin{equation}
x &=& y + z 
\end{equation}

\begin{equation}
f(x) &=& x^2
\end{equation}

\begin{equation}
f(x) &=& \sum_{k=1}^{n} xi 
\end{equation}
\begin{equation}
f(x) &=& \int \limits_{1}^{n} {xi } 
\end{equation}

\section{\ul{Exercice 2: Energy consumption model}}


\begin{equation}
E_{Tx} (l,d)  &=& 
\begin{cases}
 1E_{elec} + 1\varepsilon_{\beta}d^2,\hspace{5em}d<d_{0}  \\ 
 1E_{elec} + 1\varepsilon_{mp}d^4,\hspace{5em}d\ge d_{0}
\end{cases}
\end{equation}

\begin{equation}
\hspace{5em} E_{Tx}(1)  &=& 1E_{elec} \\
\end{equation}

\begin{equation}
\hspace{5em} d_{0}(1)  &=& \sqrt{\frac{\epsilon_{fs}}{\epsilon_{mp}}}
\end{equation}

\newpage
\section{\ul{Exercice 3: Matrice}}

\[
\begin{equation}
\begin{matrix}
   x & y  \\
   z & f  
\end{matrix}
\end{equation}
\]

\[
\begin{equation}
\begin{pmatrix}
   x & y  \\
   z & f 
\end{pmatrix}
\end{equation}
\]

\[
\begin{equation}
\begin{bmatrix}
   x & y  \\
   z & f 
\end{bmatrix}
\end{equation}
\]

\[
\begin{equation}
\begin{vmatrix}
   x & y  \\
   z & f  
\end{vmatrix}
\end{equation}
\]


\[
\begin{equation}
\begin{Vmatrix}
   x & y  \\
   z & f  
\end{Vmatrix}
\end{equation}
\]

\[
\begin{equation}
\begin{Bmatrix}
   x & y  \\
   z & f  
\end{Bmatrix}
\end {equation}
\]

\[
\begin{equation}
	\begin{bmatrix}
    a_{11} &\cdots &\cdots & a_{1n}\\
    \vdots &  a_{22} &\cdots &a_{2n}\\
		\vdots &\cdots   &  \ddots& a_{nn}
  \end{bmatrix}
\end{equation}
\]

\newpage 

\begin{figure}[!h]
\centering
\includegraphics[width=8cm]{logo.jpg} 
\caption{Logo ISNoT2018}
\end{figure}

\begin{wrapfigure}
\centering
\includegraphics[width=4cm]{logo.jpg}
\end{wrapfigure}
Logo ISNoT2018

\includegraphics[scale=0.05]{logo.jpg}
\includegraphics[scale=0.1]{logo.jpg}
\includegraphics[scale=0.2]{logo.jpg}
\includegraphics*[bb=20 20 302 334,width=3.15cm,clip]{logo.jpg}
\includegraphics*[bb=0 0 282 314 ,width=3.15cm,clip]{logo.jpg}
\newline
\newline
\newline
\begin{figure}[!h]
\centering
\includegraphics[width=12cm , height=5cm]{logo.jpg}
\caption{Logo ISNoT2018  12cm*5cm}
\end{figure}
\newline
\newline

\begin{figure}[!h]
\centering
\includegraphics[angle=45]{logo.jpg}
\caption{Logo ISNoT2018  angle=45}
\end{figure}

\newline
\newline
\begin{figure}[!h]
\centering
\includegraphics*[90,108][260,208]{logo.jpg}
\caption{[90,108][260,208]}
\end{figure}

\begin{figure}[!h]
\centering
\includegraphics[trim = 6cm 0cm 0cm 0cm, clip]{logo.jpg}
\caption{trim = 6cm 0cm 0cm 0cm}
\end{figure}





\end{document}