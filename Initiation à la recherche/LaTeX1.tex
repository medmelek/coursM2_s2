\documentclass[12 pt]{article}
\usepackage{soul}
\usepackage{ulem}
\usepackage{graphicx}
\usepackage{fancyhdr}
\usepackage{lmodern}
\usepackage{amsmath}
\usepackage{amssymb}
\usepackage{mathrsfs}

\pagestyle{fancy}


\begin{document}

\lhead{}
\chead{}
\rhead{\bfseries Oueslati Mohamed Melek}
\lfoot{Tous les tps}
\cfoot{Initiation a la recherche appliquée2}
\rfoot{\thepage}
\renewcommand{\headrulewidth}{1pt}
\renewcommand{\footrulewidth}{1pt}


\title {Table des matieres}
\maketitle

\section{Introduction}

\section{\underline{Introduction}}
\section{\underline{Definition}}
 \subsection{\uuline {advantages}}
 \subsection{\uuline{disadvantages}}
 
\section{\underline{conclusion}}



\newpage
\setcounter{section}{0} 
 { \section{\ul{Introduction}}   }

\paragraph{\qquad
 Most of the previous clustering approaches have been done aiming to balance the energy consumption of the nodes over the network. In this section, we briefly introduce some of these proposed algorithms.}
\raggedleft

\paragraph{\qquad
In Low-energy and adaptive clustering hierarchy (LEACH) [9], Energy-Efficient Unequal Clustering (EEUC) [5], Cluster Head Election mechanism using Fuzzy logic (CHEF) [6] and Multi Objectives Fuzzy Clustering Algorithm (MOFCA) [2] a probabilistic and distributed method is used, where, the node enters into a competition only if its random generated number is less than a predefined threshold value Th. For [5] [6] and [2], the competition is made by neighbors, they communicate and decide together the best node eligible to be elected as CH }\centering
\vspace{3cm}

\paragraph{\qquad
The remainder of this paper is organized as follows: Section
II discusses some of the existing clustering techniques; in
Section III the system model is presented; Section IV and V
present the proposed algorithms CHEREDC and evaluate its
performance comparing with previous algorithms. Finally, we
conclude the paper and discuss some possible future works in
Section VI.
}\raggedright
\vspace{3cm}

\newpage

\section{}
\begin{tabular}{|c|c|}
\hline
paramter & value\\

\hline
N (Number of deployed nodes) & 22 \\

\hline
R (Range of nodes in meter) & 10 \\

\hline
Eelec & 50nJ/bit \\

\hline
\epsilon fs & 10pJ/bit/m  \\

\hline
\epsilon mp & 0.001pJ/bit/mp4 \\

\hline
E (Initial energy of nodes) & 1J \\

\hline
ctrPacketLenth & 2000bits \\

\hline
PacketLenth & 4000bits \\

\hline
\end{tabular}

\section{}
\begin{tabular}{|l|l}
\hline
paramter & value\\

\hline
N (Number of deployed nodes) & 22 \\

\hline
R (Range of nodes in meter) & 10 \\

\hline
Eelec & 50nJ/bit \\

\hline
\epsilon fs & 10pJ/bit/m  \\

\hline
\epsilon mp & 0.001pJ/bit/mp4 \\

\hline
E (Initial energy of nodes) & 1J \\

\hline
ctrPacketLenth & 2000bits \\

\hline
PacketLenth & 4000bits \\

\hline
\end{tabular}


\section{}
\begin{tabular}{|r|r|}
\hline
paramter & value\\

\hline
N (Number of deployed nodes) & 22 \\

\hline
R (Range of nodes in meter) & 10 \\

\hline
Eelec & 50nJ/bit \\

\hline
\epsilon fs & 10pJ/bit/m  \\

\hline
\epsilon mp & 0.001pJ/bit/mp4 \\

\hline
E (Initial energy of nodes) & 1J \\

\hline
ctrPacketLenth & 2000bits \\

\hline
PacketLenth & 4000bits \\

\hline
\end{tabular}

\newpage

\section{}
\begin{tabular}{r l}
\hline
paramter \vline&value\\
\hline
\hline
N (Number of deployed nodes) & 22 \\

R (Range of nodes in meter) & 10 \\

Eelec & 50nJ/bit \\

\epsilon fs & 10pJ/bit/m  \\

\epsilon mp & 0.001pJ/bit/mp4 \\

E (Initial energy of nodes) & 1J \\

ctrPacketLenth & 2000bits \\

PacketLenth & 4000bits \\
\hline
\end{tabular}

\newpage
\section{Tableau II}
\begin{tabular}{|c|c|c|c|}
\hline
(XBS,YBS)& FND & HNA & LND \\
\hline
((100,100)) & 1003 & 1521 & 2693 \\
\hline
(0,0) & 119 & 1009 & 2457 \\
\hline
(250,250)  & 40 & 470 & 2063 \\
\hline
\end{tabular}

\section{Tableau II separation horizontal}
\begin{tabular}{cccc}
\hline
(XBS,YBS)& FND & HNA & LND \\
\hline
\hline
((100,100)) & 1003 & 1521 & 2693 \\

(0,0) & 119 & 1009 & 2457 \\

(250,250)  & 40 & 470 & 2063 \\
\hline
\end{tabular}


\section{Tableau II separation horizontal entre colonne}
\begin{tabular}{cccc}
\hline
\begin{tabular}{c|}(XBS,YBS) \\ \hline & \\  \end{tabular}& \begin{tabular}{c|}FND\\ \hline & \\  \end{tabular}  & HNA & LND  \\
\hline
((100,100)) & 1003 & 1521 & 2693 \\
\hline
(0,0) & 119 & 1009 & 2457 \\
\hline
(250,250)  & 40 & 470 & 2063 \\
\hline
\end{tabular}



\section{Tableau II  separation vertical }
\begin{tabular}{|c|c|c|c|}
\hline
(XBS,YBS)& FND & HNA & LND \\
\hline
((100,100)) & 1003 & 1521 & 2693 \\
\hline
(0,0) & 119 & 1009 & 2457 \\
\hline
(250,250) \vline Value & 40 & 470 & 2063 \\
\hline
\end{tabular}

\newpage 

\section{\ul{Exercise 1}}


\begin{equation}
x &=& y + z 
\end{equation}

\begin{equation}
f(x) &=& x^2
\end{equation}

\begin{equation}
f(x) &=& \sum_{k=1}^{n} xi 
\end{equation}
\begin{equation}
f(x) &=& \int \limits_{1}^{n} {xi } 
\end{equation}

\section{\ul{Exercice 2: Energy consumption model}}


\begin{equation}
E_{Tx} (l,d)  &=& 
\begin{cases}
 1E_{elec} + 1\varepsilon_{\beta}d^2,\hspace{5em}d<d_{0}  \\ 
 1E_{elec} + 1\varepsilon_{mp}d^4,\hspace{5em}d\ge d_{0}
\end{cases}
\end{equation}

\begin{equation}
\hspace{5em} E_{Tx}(1)  &=& 1E_{elec} \\
\end{equation}

\begin{equation}
\hspace{5em} d_{0}(1)  &=& \sqrt{\frac{\epsilon_{fs}}{\epsilon_{mp}}}
\end{equation}

\newpage
\section{\ul{Exercice 3: Matrice}}

\[
\begin{equation}
\begin{matrix}
   x & y  \\
   z & f  
\end{matrix}
\end{equation}
\]

\[
\begin{equation}
\begin{pmatrix}
   x & y  \\
   z & f 
\end{pmatrix}
\end{equation}
\]

\[
\begin{equation}
\begin{bmatrix}
   x & y  \\
   z & f 
\end{bmatrix}
\end{equation}
\]

\[
\begin{equation}
\begin{vmatrix}
   x & y  \\
   z & f  
\end{vmatrix}
\end{equation}
\]


\[
\begin{equation}
\begin{Vmatrix}
   x & y  \\
   z & f  
\end{Vmatrix}
\end{equation}
\]

\[
\begin{equation}
\begin{Bmatrix}
   x & y  \\
   z & f  
\end{Bmatrix}
\end {equation}
\]

\[
\begin{equation}
	\begin{bmatrix}
    a_{11} &\cdots &\cdots & a_{1n}\\
    \vdots &  a_{22} &\cdots &a_{2n}\\
		\vdots &\cdots   &  \ddots& a_{nn}
  \end{bmatrix}
\end{equation}
\]

\newpage 

\begin{figure}[!h]
\centering
\includegraphics[width=8cm]{logo.png} 
\caption{Logo ISNoT2018}
\end{figure}

\begin{wrapfigure}
\centering
\includegraphics[width=4cm]{logo.png}
\end{wrapfigure}
Logo ISNoT2018

\includegraphics[scale=0.05]{logo.png}
\includegraphics[scale=0.1]{logo.png}
\includegraphics[scale=0.2]{logo.png}
\includegraphics*[bb=20 20 302 334,width=3.15cm,clip]{logo.png}
\includegraphics*[bb=0 0 282 314 ,width=3.15cm,clip]{logo.png}
\newline
\newline
\newline
\begin{figure}[!h]
\centering
\includegraphics[width=12cm , height=5cm]{logo.png}
\caption{Logo ISNoT2018  12cm*5cm}
\end{figure}
\newline
\newline

\begin{figure}[!h]
\centering
\includegraphics[angle=45]{logo.png}
\caption{Logo ISNoT2018  angle=45}
\end{figure}

\newline
\newline
\begin{figure}[!h]
\centering
\includegraphics*[90,108][260,208]{logo.png}
\caption{[90,108][260,208]}
\end{figure}

\begin{figure}[!h]
\centering
\includegraphics[trim = 6cm 0cm 0cm 0cm, clip]{logo.png}
\caption{trim = 6cm 0cm 0cm 0cm}
\end{figure}





\end{document}